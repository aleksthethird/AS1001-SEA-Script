\section{What is this?}

Here I test what's going on.

This is the script for the 'Stars and Elementary Astrophysics' (SEA) lectures, which are part of AS1001. The script covers the \textit{skeleton of facts} for this course, including important equations and numbers. \textit{It is not a replacement for taking notes in the lectures.} The lectures will cover everything in this script, but with more illustrations, explanations, figures, animations, and examples. The slides from the lectures will \textit{not} be on Moodle. Important terms and concepts are marked with italics. Important equations are numbered. The textbook for this course is Kutner's 'Astronomy - A Physical Perspective'. 
Synopsis: SEA answers the basic question 'what is a star?'. This includes measuring distances to stars, and finding out how large, hot, and massive stars are. SEA also covers the essential toolkit for astronomers - telescopes and instruments - as well as the basics of electromagnetic radiation. \textit{SEA is the fundament for all other astronomy modules.}

\section{Introduction}

Astronomy is: the study of the stars. But astronomy also covers planets, gas clouds, galaxies, black holes, pulsars, the Universe itself - \textit{everything in the material world except things on Earth}. Astronomy needs physics, chemistry, mathematics (and perhaps biology?).

\textit{What is a star?} To the eye, stars are points of light at the night sky. They are grouped in \textit{constellations}. These are arbitrary groupings of stars - chance projections, no physical groups. Examples for well-known constellations are Orion, Ursa Major, Taurus, Cassiopeia. 

The constellations that are visible change over the year (due to the rotation of the Earth around the Sun) and over the night (due to the rotation of the Earth). 

Looking at the night sky shows: Stars have different brightnesses and colours. \textit{But what is a star really?}

\section{Distances to stars}

Fundamental problem: A point of light at the sky could be a nearby candle or a very distant supernova. To find out what stars really are, we need a method to determine the distance that is independent of the brightness. The most important method in this context is the \textit{parallax}.

\subsection{Parallax}

The parallax method is based on \textit{triangulation}. The basic principle is explained in Fig \ref{fig1}. If we can measure the angle p and the baseline s in the triangle, we can infer the distance: $\tan{(p)} = s/d$, i.e. $d = s/\tan{(p)}$. For small $p$, we can use the small angle approximation, $\tan{(tp)} \sim p$. That means:

\begin{equation}
d = s/p
\label{eq1}
\end{equation}

In astronomy, the distance Earth-Sun is used as baseline. The parallax is the apparent motion of stars on the sky caused by the rotation of the Earth around the Sun. Fig \ref{fig2} illustrates the parallax motion of a star. The parallax of stars $p$ is measured in \textit{arcseconds} (arcsec). 1" = 1 arcsec = 1/3600th of one degree. This is a very small angle.

\subsection{The parsec}

A \textit{parsec} is defined as the distance of an object that has a parallax of 1 arcsec. The semi-major axis of the Earth's orbit is $1.5 \cdot 10^{11}$\,m (defined as 1\,Astronomical Unit). If the parallax is measured in arcsec, the distance $d$ to the star from the Sun in units of parsec is simply:

\begin{equation}
d = 1/p
\label{eq2}
\end{equation}

What is 1\,pc in metric units? Start with Equ \eqref{eq1}. Use $s = 1\,\mathrm{AU} = 1.5 \cdot 10^{11}$\,m. 1\,pc distance means $p = 1" = (1/3600)$ degrees. Convert this to radians and put in Equ \eqref{eq1}: $d = 1.5 \cdot 10^{11} / 0.0000048 = 3.1 \cdot 10^{16}$\,m. 

The closest stars to the Sun are the triple system $\alpha$\,Centauri at a distance of 1.3\,pc (and a parallax of 0.75"). There are several hundred stars within 10\,pc, among them Sirius, Vega, Procyon, Altair, some of the brightest stars on the sky.

\subsection{Measuring the parallax}

How accurately can we measure positions of stars? Parallaxes are very small angles. First successful measurements in 1838-1839. Today, 0.05" can be done with ground-based telescopes. The satellite Hipparcos (1989-1993) measured parallaxes for 100.000 stars with 0.001" accuracy. The satellite Gaia (2013+) will get parallaxes for one billion stars with 0.0001" accuracy - this is still only 1\% of the stars in the Milky Way.

This means: The parallax method only covers our cosmic neighbourhood. We need other methods for objects at larger distances. There are many more methods to determine distances of stars, some will be discussed in other parts of SEA1001. But all are based on the parallax.

\section{Brightness of stars}

\subsection{Flux and luminosity}

The \textit{luminosity} $L$ is the total energy emitted by a star per seconds, i.e. it is measured in Joule/sec or Watts. On Earth we only receive a part of this energy. The \textit{flux} $f$ is the energy per second that an observer on Earth measures. \textit{Luminosity is what the source emits, flux is what the observer receives.}

\subsection{Inverse square law for the propagation of light}

Flux and luminosity are related through a basic law that describes the propagation of light in space. The light from a star spreads out \textit{isotropically} (i.e. the same amount in all directions) over the surface of a sphere (see Fig \ref{fig3}). The flux is therefore:

\begin{equation}
f = L / (4 \pi d^2)
\label{eq3}
\end{equation}

Example: Assume a detector measures $f=1$ for a star. Now we move the detector further away. At twice the distance, the light from the star has spread out to cover four times the surface, i.e. the same detector would collect 1/4 of the light. The flux drops with the inverse square of the distance.

Equ \eqref{eq3} contains three quantities, flux, luminosity and distance. The flux can be measured on Earth. The inverse square law can be used to determine luminosities, if the distance is known (from the parallax). It can also be used to determine distances, if the luminosity of a star is known. This second aspect leads to the concept of standard candles.

\subsection{Standard candles and cepheids}

A \textit{standard candle} is an astronomical object with a known brightness, i.e. we know in some way how much light it emits. From this and the measured flux the distance can be derived using the inverse square law.

\textit{Cepheids} are one example for standard candles. These are supergiant stars which vary periodically in brightness due to pulsation. Their pulsation period and their luminosity are related - the longer the period the larger the luminosity (see Fig \ref{fig4}). From the period we can derive the luminosity. From luminosity and flux follows the distance via the inverse square law.

\subsection{Units for the brightness of stars}

The unit of the flux is Watts per square meter W\,m$^{-2}$. This is very small for astronomical objects. Often used instead is the unit Jansky which is defined as $1\,\mathrm{Jy} = 10^{-26}$\,W\,m$^{-2}$\,Hz$^{-1}$

Flux is measured on a linear scale, i.e. a source of 10\,Jy is ten times brighter than a source of 1\,Jy. This is inconvenient in astronomy, more useful would be a logarithmic scale $\log{(f)}$. This leads to the concept of magnitudes.

\subsection{Magnitudes}

The unit magnitudes is derived from a system first used by the Greek astronomer Hipparcox (2nd century BC). In his catalogue of stars, 1st magnitude are the brightest stars, 6th magnitude are the stars just visible for the human eye. This system has now been adopted and extended for modern astronomy. The relation between fluxes and magnitudes is:

\begin{equation}
m_1 - m_2 = -2.5 \log{(f_1 / f_2)}
\label{eq4}
\end{equation}

Or conversely:

\begin{equation}
f_1 / f_2 = 10^{(m_1 - m_2) / -2.5}
\label{eq5}
\end{equation}

This is a logarithmic system, but with a scaling factor of 2.5. This factor means that 5\,mag difference correspond to a factor of 100 in flux. The zeropoint for the magnitude scale is Vega at $m = 0.0$.

\subsection{Magnitudes and distances}

Usually magnitudes are \textit{apparent magnitudes} (donated with small letter $m$). They relate to the flux measured on Earth and depend (as the flux does) on the distance. 

But we can also define a magnitude that relates to the luminosity and does not depend on distance. This is the \textit{absolute magnitude} (donated with capital letter $M$). The absolute magnitude is the magnitude a star would have at a distance of 10\,pc. Substituting Equ \eqref{eq3} into Equ \eqref{eq4} gives $m - M = -2.5 \log{ ((10 / d)^2) } =   -2.5 \cdot -2 \log{(d/10)}$, i.e.:

\begin{equation}
m - M = -5 \log{(d/10\,\mathrm{pc})}
\label{eq6}
\end{equation}

The quantity $m-M$ is called the \textit{distance modulus} and is another unit for distances of stars. If the distance of a star is known, the apparent magnitude $m$ can be measured and the absolute magnitude $M$ can be derived. 

\subsection{Bolometric magnitudes}

Magnitudes are usually defined for specific parts of the spectrum (e.g., $m_V$ for the visual light).  The 'bolometric magnitude' $M_{\mathrm{bol}}$ is defined as the magnitude corresponding to the total energy received from the star (i.e. the flux integrated over the full spectrum).

\section{Binary stars}

\textit{Binary stars} are two stars in mutual gravitational interaction orbiting their common center of mass (see Fig \ref{fig5}). Higher order binaries like triples and quadruples exist as well. Most stars are born as multiples.

We distinguish:
\begin{itemize}
\item{\textit{Visual binaries}: Two stars are seen separately.}
\item{\textit{Spectroscopic binaries}: Stars are not seen separately, but the spectrum shows two set of lines moving in opposite directions due to the Doppler shift (see below).}
\item{\textit{Eclipsing binaries}: Stars are not seen separately, but one star eclipses the other in regular intervals.}
\end{itemize}

Binaries are useful for two reasons: a) The orbits determined by gravitational forces, i.e. by studying them we can determine masses of stars. b) Eclipses can be used to determine the sizes of stars. Binaries are therefore important to determine fundamental properties of stars (i.e. what a star really is).

\subsection{Doppler effect}

The wavelength that an observer measures depends on the relative motion between the observer and the light source. If the source is moving towards the observer, the light is blueshifted ($\lambda < \lambda_0$). If the source is moving away from the observer, the light is redshifted ($\lambda >\lambda_0$). This is called the \textit{Doppler effect} (analogous to the Doppler effect with sound waves).

The shift in wavelength due to the Doppler effect in the light of a star is:

\begin{equation}
\Delta\lambda / \lambda = v/c
\label{eq7}
\end{equation}

Here $c$ is the speed of light, $\Delta\lambda$ is the wavelength difference $(\lambda - \lambda_0)$, and $v$ is the \textit{radial velocity} of the star, i.e. the relative speed of the star along the line of sight of the observer. Equ \eqref{eq7} is only valid for $v<<c$.

The Doppler effect shifts lines in the spectra of stars (see Sect. XXX). From these shifts and with Equ \eqref{eq7} the radial velocities of stars can be measured. The Doppler effect has many applications in astronomy; one example is the study of binary stars.

\subsection{Orbits of binaries}

Fig \ref{fig5} shows the orbit for a binary. We consider here circular orbits for simplicity. The orbital period is $P = 2 \pi r / v$. In the binary system both periods are the same, $P_1 = P_2$. From these two equations follows:

\begin{equation}
r_1 / v_1 = r_2 / v_2
\label{eq8}
\end{equation}

With the definition of the center of mass, $m_1 r_1 = m_2 r_2$, we obtain:

\begin{equation}
v_1 / v_2 = r_1 / r_2 = m_2 / m_1
\label{eq9}
\end{equation}

This means, the more massive star is orbiting at a shorter distance from the center of mass and with a smaller velocity.

\subsection{Kepler's 3rd law for binaries}

Newton's law of gravity is: $F = (G m_1 m_2) / (r_1 + r_2)^2$

In a binary system, the gravitational force equals the force to keep the star in orbit: 

\begin{equation}
m_1 v_1^2 / r_1 = G m_1 m_2 / (r_1 + r_2)^2
\label{eq10}
\end{equation}

With $v_1 = 2 \pi r_1 / P$ follows $4 \pi^2 r_1 / P^2 = G m_2 (r_1 + r_2)^2$. With $R = r_1 + r_2$ this gives after a few more steps (see textbook):

\begin{equation}
4 \pi^2 R^3 / G = (m_1 + m_2) P^2
\label{eq11}
\end{equation}

In solar system units (distance between the stars in AU, period in years, masses in solar masses), this equation simplifies to:

\begin{equation}
R^3 = (m_1 + m_2) P^2
\label{eq12}
\end{equation}

This is \textit{Kepler's 3rd law}. 

In a spectroscopic system, the radial velocities $v_1$ and $v_2$ can be measured, together with the period. This immediately gives $r_1$ and $r_2$ and thus $R$, as well as the ratio $m_1$ and $m_2$. With Kepler's 3rd law, we can calculate the sum of the masses. Together this gives individual masss for the components. 

Factors that are neglected here: elliptic orbits, inclination of orbits against the line of sight (for spectroscopic binaries) or the sky (for visual binaries). 

\subsection{Eclipsing binaries}

Fig \ref{fig6} shows the lightcurve of an eclipsing binary system. From the lightcurve, $t_1$, $t_2$, $t_3$, $t_4$ can be measured. The geometry of the system then gives expressions for the diameter $D_1$ and $D_2$ of the stars:

\begin{equation}
(t_4 - t_1) / P = (D_1 + D_2) / (2 \pi R)
\label{eq13}
\end{equation}

\begin{equation}
(t_3 - t_2) / P = (D_1 - D_2) / (2 \pi R)
\label{eq14}
\end{equation}

With these two equations the sizes of the stars can be derived in units of the orbital separation $R$. The radial velocities ($v_1$, $v_2$) and the period $P$ yield $R$, and thus the radii of the stars. This only works in an eclipsing binary which is also a spectroscopic binary. This rare type of system is therefore extremely important in astronomy.

\section{Radiation from stars}

\subsection{Electromagnetic radiation}

Summary of important facts:
\begin{itemize}
\item{EMR travels with the speed of light ($c = 3\cdot10^8$ms$^{-1}$}
\item{EMR has wave-like properties (e.g., interference).}
\item{EMR is described by a wavelength $\lambda$ or a frequency $\nu$,
with $c = \nu\lambda$.}
\item{EMR also has particle-like properties (e.g., photoelectric effect).}
\item{A 'light particle' or photon is a packet of energy with $E = h \nu = h c / \lambda$ ($h$: Planck's constant).}
\end{itemize}

\subsection{Blackbodies}

A blackbody is an idealised body in thermodynamical equilibrium with its surroundings, absorbing all radiation incident on it, and then re-radiating it. Every warm object -- including stars -- emits blackbody radiation caused by the thermal motions of its particles.

A blackbody emits some energy at all wavelengths. The spectrum of a blackbody -- the emitted flux as a function of wavelength -- is described by the Planck function shown in Fig. \ref{fig7} (in units of Wm$^{-2}$Hz$^{-1}$ster$^{-1}$).

\begin{equation}
B(\nu,T) = \frac{2h\nu^3}{c^2} \frac{1}{e^{\frac{h\nu}{kT}} - 1}
\label{eq15}
\end{equation}

At every wavelength, a hotter blackbody emits more energy than a cooler one. The peak of the Planck function shifts towards shorter wavelengths for higher temperatures. For a blackbody with temperature $T$ there is a wavelength $\lambda_{\mathrm{max}}$ at which is radiates its maximum energy, the peak of the Planck function. This is described by Wien's law. 

\begin{equation}
\lambda_{\mathrm{max}} T = 0.0029\,\mathrm{m K}
\label{eq16}
\end{equation}

Example: Sun with $T = 5800$\,K, i.e. $\lambda_{\mathrm{max}} \approx 500$\,nm

For long wavelengths and small frequencies (i.e. radio radiation), Planck's law can be approximated by the Rayleigh-Jeans law:

\begin{equation}
B(\nu,T) = 2kT \nu^2 / c^2
\label{eq17}
\end{equation}

\subsection{Stefan-Boltzmann law and luminosity}

Integrating over Equ \eqref{eq15} over all directions and frequencies yields the entire energy output of a blackbody per second per square meter, describe by the Stefan-Boltzmann law ($\sigma = 5.67$\,Wm$^{-2}$K$^{-4}$):

\begin{equation}
E(T) = \sigma T^4
\label{eq18}
\end{equation}

Note: The total energy emitted by a blackbody depends strongly on temperature.

This gives an expression for the luminosity (the total energy output) of a star, by multiplying Equ \ref{eq18} with the surface area:

\begin{equation}
L = 4 \pi R^2 \sigma T^4
\label{eq19}
\end{equation}

If two of the three quantities $L$, $R$, and $T$ are known, the third can be derived using Equ \eqref{eq19}.

\subsection{Colours of stars}

Colours for stars are measured using multi-band photometry, i.e. measuring the magnitudes in different regions of the spectrum ('bands') -- see Fig \ref{fig8}. The 'colour' is the difference between magnitudes in two bands. The colours are related to the temperatures of stars via Wien's law (Equ \eqref{eq16}).

Hot stars are bright in the blue and faint in the red, i.e. $(B-R)<0$. Cool stars are faint in the blue and bright in the red, i.e. $(B-R)>0$. Colours are defined to be 0.0 for a star like Vega.

\subsection{Kirchhoff's laws}

Stars emit \textit{continouous radiation} due to the thermal motion of particles. The continuuum is blackbody radiation, its spectrum is described by the Planck function. But stars are not pure blackbodies. Their spectra show specific regions where less light is emitted than for a blackbody. These regions appear as dark bands in the spectra and are called \textit{spectral lines}.

According to \textit{Kirchhoff's laws}, there are three types of spectra (see Fig \ref{f9}):

\begin{itemize}
\item{A continuous spectrum (light at all wavelengths) is produced by hot opaque body, usually approximated by a blackbody.}
\item{A continuous spectrum with dark absorption lines is produced by a hot opaque body seen through a transparent layer of cool gas which absorbs light at specific wavelengths. This describes a star -- the hot blackbody is seen through the cooler atmosphere of the star.}
\item{An emission line spectrum is produced by hot transparent gas. Here light is only emitted at specific wavelengths. This can occur, for example, when clouds of gas are heated by nearby hot stars.}
\end{itemize}

\subsection{Origin of spectral lines}

Spectral lines are caused by absorption or emission of EMR in atoms or molecules in the atmospheres of stars. An atom can absorb or emit only photons of certain $\lambda$, when electrons 'jump' from one energy level to another. The change in energy is proportional to the frequency of the emitted or absorbed photon:

\begin{equation}
E = h \nu = h c / \lambda
\label{eq20}
\end{equation}

The energy differences between the electron levels in atoms are very small, typically in the range of electron Volts (1\,eV is $1.602 \cdot 10^{-19}$\,J).

\subsection{Spectral lines in Hydrogen}

Hydrogen is the simplest atom, with only one proton and one electron. The energy levels for the Hydrogen electron are shown in Fig \ref{f10}.

The energy difference between ground state ($n=1$) and first excited state  ($n=2$) is $E_{12} = 10.2$\,eV. With Equ \eqref{eq20} this gives $\lambda_{12} = 121.6$\,nm. This line is called \textit{Lyman $\alpha$} and is in the ultraviolet part of the spectrum. Other transitions to or from the ground state have higher energies, i.e. shorter wavelengths. The Lyman limit at $\lambda = 91.2$\,nm corresponds to the energy required to move an electron from the ground state to 'infinity', i.e. out of the Hydrogen atom.

In general, the wavelength for the photons corresponding to the transitions in the Hydrogen atom are described by the \textit{Rydberg equation}:

\begin{equation}
1/\lambda = R (1/n_l^2 - 1/n_u^2)
\label{eq21}
\end{equation}

Here, $R$ is the Rydberg constant ($R = 1.097 \cdot 10^{-7}$\,m$^{-1}$). $n_l$ and $n_u$ are the quantum numbers of the lower and upper level in a specific transition.

Lyman series: $n_l = 1$, $n_u >1$, in the UV
Balmer series: $n_l = 2$, $n_u >2$, in the optical
Paschen series: $n_l = 3$, $n_u >3$, in the infrared

\subsection{Equivalent widths}

The \textit{equivalent width} is a quantity that expresses how strong an absorption line is. In Fig \ref{fig11} the basic principle is illustrated. The equivalent width is the area of a rectangle, from zero to the continuum level, that has the same area as the absorption line. Note that the equivalent width is measured in units of wavelength.

\subsection{Spectra and temperature}

Stars show a wide diversity of spectra -- some have very few lines, others have millions. The appearance of the spectrum depends on the configuration of the electrons in the atoms and molecules in the atmospheres of stars. This, in turn, depends strongly on the surface temperature of stars. The temperature is therefore the parameter that determines essentially which spectral lines appear in a given star. Or, conversely, from the set of lines in the spectra and their respetive strengths the temperature can be inferred (in comparison with laboratory measurements).

Hot stars (10000-40000\,K) have mostly ionised gas in the atmosphere, they only show lines of Helium and Hydrogen. Sun-like stars ($\sim 6000$\,K) have mostly neutral gas, their spectra are dominated by lines caused in metal atoms. Cool stars (3000-4000\,K) have molecules and show broad absorption bands, mostly due to metal oxides. See the textbook for a selection of stellar spectra.

\subsection{Spectra and luminosity}

Stars are divided in five luminosity classes denoted by Roman numerals: Dwarfs are the most common. Giants (III) are common as well. Subgiants (IV) are between dwarfs and giants are subgiants (IV). The bright giants (II) and supergiants (I) extend the scheme to larger luminosities. 

Luminosity affects spectra in subtle ways, for example via the surface gravity ($g = GM/R^2$). Giants are larger than dwarfs at the same temperature (see Equ \eqref{19}), hence they have lower surface gravity. This means that the pressure at the surface is lower because $p \propto g$. The width of the absorption lines are broadened by pressure, i.e. a giant star has narrower lines than a dwarf star. In principle, this effect allows us to estimate the size of stars and their surface gravity from the shape of spectral lines.

\subsection{Two-dimensional spectral classification}

The appearance of the absorption line spectrum determines the star's \textit{spectral type}. The spectral type is an observable property, it is related to colour and temperature. This is the first dimension of the spectral classification. 

The current spectral typing scheme (called 'MK system') includes a sequence of 7 fundamental types, denoted O, B, A, F, G, K, M. Each type is divided in 10 subtypes, for example A1, A2, ..., A9. Hot blue stars have spectral type OB. Cool, blue stars have spectral type KM. Recently three new spectral types have been added to include ultracool dwarfs (L, T, Y).

In addition to the spectral type, stars are given a Roman numeral denoting the luminosity class. This yields a two-dimensional classification scheme, called \textit{MK system} (after Morgan \& Keenan). 

Examples: the Sun G2V, Vega A0V, Aldebaran K5III

This scheme describes many 'normal' stars well. Some exceptions are covered by additional letters, for example, M5Ve for stars with emission lines.

\section{The Hertzsprung-Russell diagram}

The \textit{Hertzsprung-Russell diagram} (HRD) is one of the most fundamental figures in astrophysics (see textbook for examples). When plotting luminosity over temperature, most normal stars are located along a well-defined area of the diagram, called the \textit{main sequence}, which stretches from cool, faint stars to hot, luminous stars. To the right side of the main sequence are the giants (cool, large, luminous stars). Below the main sequence are white dwarfs (hot, small, faint stars).

The HRD can be plotted in a number of ways. Two of the most common ones are a) luminosity vs. temperature or b) absolute magnitude vs. colour/spectral type (see Fig \ref{f12}).

The HRD is the fundament for all theories of stellar evolution and will be discussed more in detail in other parts of AS1001 and other courses. The pre-requisite for the HRD is the measurement of stellar luminosities and temperatures. 

\section{Motions of stars}

There are many reasons why stars move on the sky. We already know about the parallax, which is an apparent motion due to the motion of the Earth around the Sun. Stars in binary systems move on their orbits. Stars also move in wide orbits around the center of the Milky Way. In addition, stars can have quite random motions.

\subsection{Sky coordinates}

The equatorial coordinates are a project of the Earth's latitude and longitude on the celestial sphere. This is illustrated in Fig \ref{f13}.

Longitude is called \textit{right ascension} $\alpha$ and is usually counted in hours from 0 to 24. The zeropoint for the right ascension is the vernal equinox. 

Latitude is called \textit{declination} $\delta$ and is usually counterd in degrees from 0 at the equator to $\pm 90$ at the poles. 

\subsection{Measuring motions of stars}

The three-dimensional motion of stars in space is described by three components:

a) The \textit{radial velocity} $v_r$ is the component along the line of sight of the observer (i.e. perpendicular to the plane of the sky). The radial velocity can be measured via the Doppler effect (see Sect. XXX). Typical radial velocities for stars in our Galaxy are $<100\,$km/s, i.e. much smaller than the speed of light. Larger radial velocities and large Doppler shifts are measured for distant galaxies.

b) The \textit{proper motion} $\mu$ is the motion of the star in the plane of the sky. The proper motion can be measured from repeated images of the same object in arcsec/yr. Typical proper motions are fractions of 1\,arcsec per year, even for the nearest stars. The proper motion has two components, one in right ascencion, one in declination: 

\begin{equation}
\mu = \sqrt{(\mu_\alpha \cos{\delta])^2 + \mu_\delta^2}
\label{eq22}
\end{equation}

The proper motion can be used to derive the transverse velocity components, i.e. the velocity in the plane of the sky, if the distance $d$ is known. With small angle approximation (see Sect XXX), the path of the star over a year is $x = \mu d$. With $\mu$ in arcsec/yr and $d$ in parsec, the transverse velocity $t$ in km/s is:

\begin{equation}
t = 4.74 \mu d
\label{eq23}
\end{equation}

The full space motion of stars is:

\begin{equation}
s = \sqrt{v_r^2 + t^2}
\label{eq24}
\end{equation}

\section{Telescopes}

Telescopes collect EMR from astronomical sources. We use telescopes for two reasons: a) They have a bigger \textit{collecting area} than the human eye, i.e. we can see fainter sources. b) They increase the \textit{angular resolution} compared with the human eye, i.e. we can see more details and measure more accurate positions.

An astronomical telescope is a system that brings parallel rays of light from a distant source to a \textit{focal plane}.

\subsection{Telescope types}

There are two fundamental types:

a) \textit{Refractors} use lenses to focus the light (objective lense and eyepiece). Largest refractor: 1.0\,m Yerkes telescope from 1888. 

b) \textit{Reflectors} use mirrors (at least one, usually more than one) to focus the light. Largest reflectors: 
11\,SALT in South Africa, 2x10\,m Keck in Hawaii, 4x8\,m VLT in Chile. Reflectors come in a variety of designs. Some of the most common ones are shown in Fig \ref{f15}.

\textit{Catadioptric telescopes} are telescopes that combine lenses and mirrors. An example is the Schmidt telescope which has the capability to make images with a very wide field of view. The James Gregory Telescope in St Andrews is a catadioptric Schmidt-Cassegrain telescope.

\subsection{Reflector optics}

The path of light through a refractor is shown in Fig \ref{f14}. Important concepts are the \textit{focal plane} (the plane where the light is focused) and the \textit{effective focal length} $f_e$ (the distance between objective lense and focal plane.

The focal ratio $f$ of a telescope is defined as $f = f_e/D$ (with $D$ being the diameter of the lens/mirror). The focal ratio describes how fast the beam converges to the focal plane. The focal ratio is usually written in units of the aperture. Example: A telescope with $D = 1$\,m and $f_e = 3$\,m has a focal ratio of $f/3$. 

Small $f/$ means small image scale, but fast light collection (light is concentrated into a small area in the focal plane). This is ideal for wide-field surveys. Large $f/$ means large image scale, but slow light collection. This is ideal for seeing details and measuring accurate positions.

\subsection{Diffraction limit}

The \textit{resolution} of a telescope is the minimum angular separation of two sources on the sky, that can be seen separately. 

The resolution is limited by the diffraction of light around the edges of the optics in the telescope. For a circular aperture, the diffraction causes each source of light to appear as a series of concentric rings in the focal plane (\textit{Airy pattern}). \textit{Rayleigh's criterion} provides an equation for the \textit{diffraction limit} of a telescope:

\begin{equation}
\alpha = 1.22 \lambda / D\,\mathrm{radians} \sim 2.5 \cdot 10^5 \lambda / D\,\mathrm{arcsec}
\label{eq22}
\end{equation}

Here, $D$ is the diameter of the aperture of the telescope (i.e. the diameter of the objective lens or the main mirror). Examples for observations at optical wavelength ($\lambda = 500$\,nm): $\alpha = 1$\,arcsec for $D = 0.125$\,m. $\alpha = 0.03$\,arcsec for 4-m telescope.

\subsection{Telescopes at different wavelengths}

\textbf{X-ray domain:} $\lambda$ of 0.01 - 10\,nm

Problem: These wavelengths are comparable to the distance between the atoms, i.e. the photons 'see' a rough surface. 

The solution is grazing reflection -- X-ray telescopes use nested rings of highly polished mirrors (see Fig \ref{f15}).

\textbf{Infrared:} $\lambda$ of 1-500\,$\mu m$

Problem: Water vapour in the atmosphere absorbs most infrared radiation, but observations are possible at high altitude in specific wavelength windows or from space. Also, since everything that is warm emits infrared radiation, the cooling of the instrument becomes very important. 

Examples of infrared satellites: IRAS, ISO, Spitzer, Herschel, WISE

\textbf{Radio telescopes:} $\lambda$ from mm to m

Problem: At long wavelengths the diffraction limit is very large. Put $\lambda = 20$\,cm in Equ \eqref{eq22} -- an aperture of 50\,km would be needed for a resolution of 1\,arcsec.

The largest single dish radio telescope is Arecibo with $D = 305$\,m

The solution is \textit{interferometry}. In short, interferometers combine the power of multiple telescopes. Many telescopes with a separation $B$ have the same resolution as one single telescope with aperture $B$ - for the resolution only the distance between the edges of the telescope counts.

Radio interferometers are built with baselines ranging from a few hundred meters (for example, the American Very Large Array) to thousands of kilometers (for example, the European LOFAR array). The principle of interferometry is also used for optical and infrared radiation.

\section{Observations with astronomical telescopes}

Observations from the ground are affected by the atmosphere of the Earth in several ways.

\subsection{Seeing}

Stars appear blurred in images from the ground because of turbulent motions of the air along the line of sight which rapidly change the refraction index of the air. In typical images with exposure times of seconds or more, stars therefore appear as a 'disk' -- the superposition of many different 'speckles'. The diameter of this disk is called the \textit{seeing}. 

Typically the seeing for good sites is 0.5-1\,arcsec; in St Andrews we reach 2\,arcsec. Compare this with the diffraction limit -- for typical astronomical observations with telescope aperture $\lt 10$\,cm, the seeing determines the resolving power of a telescope.

\subsection{Extinction}

Extinction is the absorption and scattering of light along the line of sight by molecules in the atmosphere. The extinction depends strongly on wavelength. The atmosphere is almost transparent in the optical wavelength domain (300-800\,nm) and in the radio (1\,cm to 20\,m). In addition there are smaller windows in the infrared. In the IR most of the absorption is due to water vapour. Everywhere else the atmosphere absorbs all the light (for example, UV and X-ray radiation).  

\subsection{Sites for astronomical telescopes}

When choosing the site for a big telescope, a number of factors needs to be considered:

\begin{itemize}
\item{weather: many clear nights desirable, i.e. above main cloud layer ($\lt 2000$\,m), very little rainfall}
\item{seeing}
\item{light pollution}
\item{low humidity}
\item{accessible}
\item{political factors}
\end{itemize}

Most big telescopes are built in dry areas on the top of mountains. Examples: La Palma, Mauna Kea in Hawaii, Cerro Paranal in the Atacama desert in Chile

\section{Instruments}

Images are used to study structures of galaxies or nebulae, to measure brightnesses of stars (photometry), or to measure positions of objects (astrometry).

Before 1980s, the typical imaging device was the \textit{photographic plate}. Digitised versions of photographic plate material are still in use (for example, the Digital Sky Survey).

Nowadays, digital cameras are widespread in astronomy. For optical observations, \textit{CCDs} (charge-coupled devices) are used. CCDs are more sensitive than photographs, particularly in the red. They constitute a grid of 1000's of pixels, typical pixel size is $\sim 15\,\mu m$.

Spectra are a record of the emitted light as a function of wavelength. They are measured with a \textit{spectrograph}. The light from the telescope enters the spectrograph through a slit, then is dispersed on a grism or a prism, before being recorded with a camera.

An important parameter of spectrographs is the spectral resolution $R$:

\begin{equation}
R = \lambda / \Delta\lambda
\label{eq23}
\end{equation}

$R$ depends on slit width, grating properties, pixel size, seeing, etc. 

For example: A Doppler shift of 10\,km/s needs to be measured. What spectral resolution is needed? Combining Eq \ref{eq7} and Eq \ref{eq23} we obtain: $R = \lambda / \Delta\lambda = c / v  = 30000$. For $\lambda = 450$\,nm, this corresponds to a wavelength shift of $\Delta\lambda = 0.015$\,nm (which is very small).

\section{Synopsis}

\subsection{Properties of stars}

\textbf{Observable properties:} position, flux, magnitude, colour, spectrum, spectral type, lightcurve

\textbf{Physical properties:} luminosity, velocity, absolute magnitude, size, mass, temperature

\textit{Astronomy means finding clever methods to get physical properties from observable properties.}

\subsection{What is a star?}


