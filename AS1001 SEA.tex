\section{What is this?}

This is the script for the 'Stars and Elementary Astrophysics' (SEA) lectures, which are part of AS1001. The script covers the \textit{skeleton of facts} for this course, including important equations and numbers. \textit{It is not a replacement for taking notes at the lectures.} The lectures will cover everything in this script, but with more illustrations, explanations, figures, animations, and examples. The slides from the lectures will \textit{not} be on Moodle. Important terms and concepts are marked with italics. Important equations are in bold. The textbook for this course is Kutner's 'Astronomy - A Physical Perspective'. 

Synopsis: SEA answers the basic question 'what is a star?'. This includes measuring distances to stars, and finding out how large, hot, massive stars are. SEA also covers the essential toolkit for astronomers - telescopes and instruments - as well as the basics of electromagnetic radiation. \textit{SEA is the fundament for all other astronomy modules.}

\section{Introduction}

Astronomy is: the study of the stars. But astronomy also covers planets, gas clouds, galaxies, black holes, pulsars, the Universe itself - so, everything in the material world except things on Earth.

Astronomy needs physics, chemistry, mathematics (and perhaps biology?).

\textit{What is a star?} To the eye, stars are points of light at the night sky. They are grouped in \textit{constellations} - these are arbitrary groupings of stars, which means: chance projections, no physical groups. Today only meaningful to 'find your way' on the sky. Examples for well-known constellations are Orion, Ursa Major, Taurus, Cassiopeia. 

The constellations that are visible change over the year (due to the rotation of the Earth around the Sun) and over the night (due to the rotation of the Earth). 

Looking at the night sky shows: stars have different brightnesses, colours, positions. \textit{But what is a star really?}

\section{Distances to stars}

Fundamental problem: A point of light at the sky could be a nearby candle or a very distant supernova. To find out what stars really are, we need a method to determine the distance that is independent of the brightness. The most important method in this context is the \textit{parallax}.

\subsection{Parallax}

The parallax method is based on \textit{triangulation}. The basic principle is explained in Fig. 1. If we can measure the angles and the baseline in the triangle ABC, we can infer the distance: $\tan{(p)} = s/d$, i.e. $d = s/\tan{(p)}$. For small $p$, we can use the small angle approximation, $\tan{(tp)} \sim p$. That means:

\begin{equation}
d = s/p
\end{equation}

In astronomy, $p$ (the parallax) is measured in \textit{arcseconds} (arcsec). 1" = 1 arcsec = 1/3600th of one degree. This is a very small angle.

\subsection{The parsec}

A parsec is defined as the distance of an object that has a parallax of 1 arcsec. Fig. 2 illustrates the idea. The semi-major axis of the Earth's orbit is $1.5 \cdot 10^{11}$\,m (defined as 1\,Astronomical Unit). If the parallax is measured in arcsec, the distance $d$ to the star from the Sun in units of parsec is simply:

\begin{equation}
d = 1/p
\end{equation}

What is 1\,pc in metric units? Start with Equ 1, $d = s/p$. Use $s = 1\,\mathrm{AU} = 1.5 \cdot 10^{11}$\,m. 1\,pc distance means $p = 1" = (1/3600)$ degrees. Convert this to radians and put in Equ 1: $d = 1.5 \cdot 10^{11} / 0.0000048 = 3.1 \cdot 10^{16}$\,m. 

The closest stars to the Sun are at 1.3\,pc (and a parallax of 0.75" the $\alpha$\,Centauri system (a triple star). There are several hundred stars within 10\,pc, among them Sirius, Vega, Procyon, Altair, some of the brightest stars on the sky.

\subsection{Measuring the parallax}

Critical question: How accurately can we measure positions of stars? Parallaxes are very small angles! First successful measurements in 1838-1839. Today, 0.05" can be done with ground-based telescopes. The satellite Hipparcos (1989-1993) measured parallaxes for 100.000 stars with 0.001" accuracy. The satellite Gaia (2013+) will get parallaxes for one billion stars with 0.0001" accuracy - this is still only 1\% of the stars in the Milky Way.

This means: The parallax only covers our cosmic neighbourhood. We need other methods for objects at larger distances. There are many more methods to determine distances of stars, some will be discussed in other parts of SEA1001. But all are based on the parallax.

\section{Brightness of stars}

\subsection{Flux and luminosity}

The \textit{luminosity} $L$ is the total energy emitted by a star per seconds, i.e. it is measured in Joule/sec or Watts. On Earth we only receive a part of this energy. The \textit{flux} $f$ is the energy per second that an observer on Earth measures. \textit{Luminosity is what the source emits, flux is what the observer receives.}

\subsection{Inverse square law for the propagation of light}

Flux and luminosity are related through the basic law that describes the propagation of light. The light from a star spreads out isotropically (i.e. the same amount in all directions) over the surface of a sphere(see Fig. 3). The flux is therefore:

\begin{equation}
f = L / (4 \pi d^2)
\end{equation}

Example: Assume a detector measure f=1 for a star. Now we move the detector further away. At twice the distance, the light from the star has spread out to cover four times the surface, i.e. a light detector with a fixed area collects 1/4 of the light. The flux drops with the inverse square of the distance.

Equ 3 contains three quantities, flux, luminosity and distance. The flux can be measured. The inverse square law can be used to determine luminosities, if the distance is known (from the parallax). can be used to determine distances, if the luminosity of a star is known. It can also be used to determine distances, if the luminosity of a star is known. This second aspect leads to the concept of standard candles.

\subsection{Standard candles and cepheids}

A standard candle is an astronomical object with a known brightness, i.e. we know in some way how much light it emits. From this and the measured flux the distance can be derived using the inverse square law. Standard candles usually have properties that do not vary with distance.

Cepheids are one example for standard candles. These are supergiant stars which vary in brightness due to pulsation. Their pulsation period and their luminosity are related - the longer the period the larger the luminosity (see Fig. 4). From the period and the flux the distance can be derived.

\subsection{Units for the brightness of stars}

The unit of the flux is Watts per square meter. This is very small for astronomical objects. Often used instead: unit 'Jansky' which is defined as $1\,Jy = 10^{-26}$\,W\,m$^{-2}$\,Hz${-1}$

