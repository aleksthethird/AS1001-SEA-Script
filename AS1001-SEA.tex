\textbf{What is this?}

This is the script for the 'Stars and Elementary Astrophysics' (SEA) lectures, which are part of AS1001. The script covers the \textit{skeleton of facts} for this course, including important equations and numbers. \textit{It is not a replacement for taking notes at the lectures.} The lectures will cover everything in this script, but with more illustrations, explanations, figures, animations, and examples. The slides from the lectures will \textit{not} be on Moodle.
 
The textbook for this course is Kutner's 'Astronomy - A Physical Perspective'. It is also used for 2nd year astronomy and explains things different, which makes it very useful. 

Important terms and concepts are marked with italics. Important equations are in bold.

Synopsis: SEA answers the basic question 'what is a star?' - which includes, measuring distances to stars, and finding out how large, hot, massive stars. SEA also covers the toolkit for astronomers - telescopes and instruments - as well as the basics of electromagnetic radiation.

\section{Introduction}

Astronomy is: the study of the stars. But astronomy also covers planets, gas clouds, galaxies, black holes, pulsars, the Universe itself - so, everything in the material world except things on Earth.

Astronomy needs physics, chemistry, mathematics (and biology in the future?).

\textit{What is a star?} To the eye, stars are points of light at the night sky.

They are grouped in \textit{constellations} - these are arbitrary groupings of stars, which means: chance projections, no physical groups. Today only meaningful to 'find your way' on the sky. Examples for well-known constellations are Orion, Ursa Major, Taurus, Cassiopeia. 

The constellations that are visible change over the year (due to the rotation of the Earth around the Sun) and over the night (due to the rotation of the Earth). 

Looking at the night sky shows: stars have different brightnesses, colours, positions. 

\textit{But what is a star really?}

\section{Distances to stars}


